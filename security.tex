\section{Security}

\subsection{Graded DDH}
In this section we instantiate the Graded Decisional Diffie-Hellman assumption for our scheme, introduced in \cite{ggh13a} as the graded encoding scheme analogue of the multilinear decisional Diffie-Hellman assumption in \cite{bs}.  For a full definition of graded encoding schemes, refer to Appendix A.

\begin{definition}[$\kappa$-GDDH]
\label{kgddh}
For a security parameter $\lambda \in \mathbb{N}$ the $\kappa$-Graded Decisional Diffie-Hellman assumption is to distinguish between the distributions $\mathcal{D}_0[(\mathsf{pp}, \pzt)]$ and $D_1[(\mathsf{pp}, \pzt)]$ for $(\mathsf{pp}, \pzt) \leftarrow \mathsf{setup}(1^\lambda, 1^\kappa)$ where: 

$$\mathcal{D}_0[(\mathsf{pp}, \pzt)] = \left\{ \left(\{C_i'\}_{i\in [\kappa + 1]}, \gamma \right) \;\middle|\; \begin{array}{ll}
\forall i \in [\kappa + 1], &  U_i \leftarrow \mathsf{samp}(\mathsf{pp}),\\
& C_i \leftarrow \mathsf{enc}(\mathsf{pp}, 1, U_i)\\
& C_i' \leftarrow \mathsf{rerand}(\mathsf{pp}, C_i)\\
\multicolumn{2}{l}{\gamma \leftarrow \mathsf{ext}(\mathsf{pp}, \pzt, U_{\kappa+1} \cdot\prod_{i \in [\kappa]} C_i)}
\end{array}\right\}$$
and
$$\mathcal{D}_1[(\mathsf{pp}, \pzt)] = \left\{ \left(\{C_i'\}_{i\in [\kappa + 1]}, \gamma \right) \;\middle|\; \begin{array}{ll}
\forall i \in [\kappa + 1], &  U_i \leftarrow \mathsf{samp}(\mathsf{pp}),\\
& C_i \leftarrow \mathsf{enc}(\mathsf{pp}, 1, U_i)\\
& C_i' \leftarrow \mathsf{rerand}(\mathsf{pp}, C_i)\\
\multicolumn{2}{l}{\gamma \leftarrow \{0,1\}^\lambda}
\end{array}\right\}$$
\end{definition}

We conjecture that this problem is hard for our scheme.

\subsection{Overview}

Let $n$, $q$ and $\chi$ be set such that the LWE$_{n,q,\chi}$ assumption holds. 

The security of our scheme relies on hiding $\vec{v}$, $z$ and $T$; with these parameters an adversary could decrypt an encoding at any level.

\subsection{Security of the GSW encryptions}   
We can view (fresh) GSW encryptions as indistinguishable from random binary matrices due to Lemma 1 in \cite{gsw}.

\subsection{Zeroizing attacks, SubM, DLIN}
Attacks on previous schemes \cite{chl,cgh,hj} that make use of encodings of 0 fail against our scheme for a number of reasons. The principal advantage that our scheme has over others is that zero-tested encodings of $0$ are not exact multiples of noise elements, and instead result in LWE-esque instances with some additive noise.  To illustrate, a zero-tested encoding of $0$ in \cite{clt} or \cite{ggh13a} is of the form $hr$ where $h$ is a ``small-ish" noise element, and $r$ is the noise present in the encoding.  In our scheme, however, a zero-tested encoding of $0$ is of the form
$$\left<\w, C \u\right> = \vec{1}\cdot \bdfn(U)\cdot (\vec{e}_C + C_\mu\vec{\epsilon})$$
where $\vec{e}_C$ is the noise present in the encoding (the $r$ from before).  Note that this is essentially the LWE-esque instance $C_\mu\vec{\epsilon} + \vec{e}_C$ (except $C_\mu$ is unknown to the adversary, unlike in typical LWE instances) multiplied by a uniform binary matrix with the resulting entries summed by the row vector $\vec{1}$.  In particular, the noise in the encoding is $\vec{e}_C$, but we have the additive term $C_\mu\vec{\epsilon}$ which breaks the previous attacks that rely on the exact multiples for things like computing GCDs and determinants. 

The most recent CLT construction \cite{clt15} changes the zero-testing parameter so that zero-tested elements also have some additive term $ax_0$ over the integers, due to the hidden modulus $x_0$, so that the attack fails in a similar way.  For that reason the authors conjecture that the subgroup membership (SubM) and decision linear (DLIN) problems are hard in their new version, as the attacks that break those assumptions in the old CLT and GGH no longer apply.  Similarly, we believe those problems to be hard in our map.

\subsection{The homomorphism of $\flattenfn$}

It is helpful to note that the $\flattenfn$ function does not preserve all algebraic structure with respect to $\powfn$ vectors.  For example, matrix multiplication does not work in this sense:
\begin{align*}
\flattenfn(AB)\cdot \powfn(\vec{v}) &= AB \cdot \powfn(\vec{v})\\
&\not= \flattenfn(A)\cdot \flattenfn(B)\cdot \powfn(\vec{v})\\ 
&= \flattenfn(A)\cdot B\cdot \powfn(\vec{v})
\end{align*}
Thus we would expect the use of this function to only reduce the number of possible algebraic attacks.

\subsection{Attacks on the zero-testing element}

We first want to rule out that we cannot get useful information out of the zero-testing elements themselves.  Note that our zero-testing elements are the following vectors:
\begin{align*}
\u &= z^\kappa T\cdot (\vec{v} + \vec{\epsilon})\\
\w &= (1,1,\ldots, 1)\cdot\bdfn(U)\cdot T^{-1}
\end{align*}
where $z^\kappa$, $T$, and $U$ are uniformly random elements.  Multiplying them together we get:
$$z^\kappa (1,1,\ldots, 1)\cdot (U\vec{s} + \bdfn(U)\cdot\vec{\epsilon})$$

\subsection{Getting parameters from zero-tested encodings}

Though we have avoided the zeroizing attacks of \cite{chl, cgh}, we now draw our attention to using the zero-testing element in breaking the scheme.  A zero-tested encoding at level-$i$ will look like the following, where $\vec{e}_C$ is the error in the GSW encryption after some number of operations:
$$\left<\w, C \u\right> = z^{\kappa-i}\vec{1}\cdot(\mu(U \vec{s}) + \bdfn(U)\cdot (\vec{e}_C + C_\mu\vec{\epsilon}))$$
This is a standard ``LWE instance" in the case that $\mu \not= 0$ as $U$ is a uniformly random matrix.  However, in the case that $\mu = 0$ and $i = \kappa$ we have:
\begin{equation}\label{eq:topleveltested}
\left<\w, C \u\right> = \vec{1}\cdot \bdfn(U)\cdot (\vec{e}_C + C_\mu\vec{\epsilon})
\end{equation}
We can view this as an LWE instance where the ``public" key $C_\mu$ is the product of $\kappa$ binary matrices that we can think of as indistinguishable from uniform (as they are GSW encryptions); where the secret is drawn from the discrete Gaussian $\mathcal{D}_{\Z^n, \sigma'}$; and where the error $\vec{e}_C$ is drawn from the discrete Gaussian $\mathcal{D}_{\Z^n, \sigma}$.  We then essentially take a random subset sum of this LWE vector by pre-multiplying by $\vec{1}\cdot\bdfn(U)$.

\subsection{RLWE variant}
We could also work over the ring $\Rq = \R/q\R \cong \Zq/\langle x^n + 1 \rangle$ instead of $\Zq$ (and let $N = 2\ell$ instead of $N = (n+1)\ell$).  We then could alter the zero-testing vector $\u = T\cdot(\alpha\vec{v} + \vec{\epsilon})$ for some small ring element $\alpha \in \Rq$.  This would then make zero-tested encodings of $0$ look like:
$$\vec{1}\cdot \bdfn(U)\cdot (\alpha\vec{e}_C + C_\mu\vec{\epsilon}$$
This would then make the resulting value a ring element that looks like the RLWE-esque instance $\alpha r + e$.

