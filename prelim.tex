\section{Preliminaries}

First, let $\R = \mathbb{Z}[x] / \left<x^n + 1\right>$ where $n$ is a power of 2, and let $\Rq = \R/q\R \cong \mathbb{Z}_q[x] / \left<x^n + 1\right> $ be the ring that contains $\ell = \lceil\log q\rceil$ bit coefficients.  For an element $f \in \R$, let $\norm{k}{f}$ denote the $\ell_k-$norm of the vector representation of $f$ in $\mathbb{Z}^n$ for all $k \in \mathbb{N}$, where $\norm{\infty}{f}$ is the $\ell_\infty$-norm (the largest coefficient of $f$).  For a vector $\vec{v} \in \R^N$, let $\norm{k}{\vec{v}}$ denote the largest element of $\vec{v}$ under the $\ell_k$-norm.

\subsection{Bit operations}

The GSW paper outlines a few operations on vectors that we extend to the ring setting, as in \cite{shield}.  For some $k$, let $\ell = \lceil \log q \rceil$ and $N = k\ell$, and let $\vec{a}$ be a $k$-dimensional vector of ring elements in $\Rq$.  Then, the bit-decomposition function $\bdfn: \Rq^k \to \Rq^N$ is given by $$\bdfn(\vec{a}) = (a_{1,0}, \ldots, a_{1, \ell-1}, \ldots, a_{k, 0}, \ldots, a_{k, \ell-1})$$ where $a_{i,j}$ is the polynomial whose entries are the $j$-th bit each coefficient in the $i$-th entry of $\vec{a}$.  That is, we decompose a vector $\vec{a}$ that has polynomials with $\ell$-bit coefficients into an $N$-dimensional vector with polynomials of 0/1 coefficients.  $\bdifn: \Rq^N \to \Rq^k$ is the inverse of this operation.  Specifically, for a vector $\vec{a} = (a_{1,0}, \ldots, a_{1, \ell-1}, \ldots, a_{k, 0}, \ldots, a_{k, \ell-1}) \in \Rq^N$, we have $$\bdifn(\vec{a}) = \left(\sum 2^j a_{1,j}, \ldots, \sum 2^j a_{k, j}\right)$$  Note that we define this function for any vector $\vec{a} \in \Rq^N$, not just vectors with 0/1 polynomials (which is what $\bdfn$ produces).  Next, for a vector $\vec{a} \in \Rq^N$, define $\flattenfn: \Rq^N \to \Rq^N$ composing these functions: $$\flattenfn(\vec{a}) = \bdfn(\bdifn(\vec{a}))$$  Finally, define $\powfn: \Rq^k \to \Rq^N$ by $$\powfn(\vec{a}) = (a_1, 2a_1, \ldots, 2^{\ell-1}a_1, \ldots, a_k, 2a_k, \ldots, 2^{\ell-1}a_k)$$  As in GSW, the important properties of these operations are:

\begin{itemize}
\item $\left< \bdfn(\vec{a}), \powfn(\vec{b}) \right> = \left< \vec{a}, \vec{b} \right>$
\item $\left< \flattenfn(\vec{a}), \powfn(\vec{b}) \right> = \left< \vec{a}, \powfn(\vec{b}) \right>$
\end{itemize}

We extend these operations to matrices by performing them on each row.  Thus, for matrices $A \in \Rq^{N \times k}$, $C \in \Rq^{N \times N}$ and a vector $\vec{b} \in \Rq^k$ we have:

\begin{itemize}
\item $\bdfn(A)\cdot\powfn(\vec{b}) = A\cdot\vec{b}$
\item $\flattenfn(C)\cdot\powfn(\vec{b}) = C\cdot\powfn(\vec{b})$
\end{itemize}

The $\flattenfn$ function keeps our encodings small while maintaining correctness with the zero-tester, which is a result of the $\powfn$ function.  In our scheme we set $k = 2$ so that $N = 2\ell$.