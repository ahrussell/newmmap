\section{Preliminaries}\label{sec:prelim}

For a vector $\vec{v} \in \Z^N$, let $\norm{k}{\vec{v}}$ denote the $\ell_k$-norm of the vector for any $k \in \mathbb{N}\cup \{\infty\}$. For any $m \in \mathbb{N}$ we denote the set $\{1,2,\ldots,m\}$ by $[m]$.

\subsection{Bit operations}

The GSW paper outlines a few operations on vectors, as in \cite{shield}.  For some $k$, let $\ell = \lceil \log q \rceil$ and $N = k\ell$, and let $\vec{a}$ be a $k$-dimensional vector of ring elements in $\Zq$.  Then, the bit-decomposition function $\bdfn: \Zq^k \to \Zq^N$ is given by $$\bdfn(\vec{a}) = (a_{1,0}, \ldots, a_{1, \ell-1}, \ldots, a_{k, 0}, \ldots, a_{k, \ell-1})$$ where $a_{i,j}$ is the polynomial whose entries are the $j$-th bit each coefficient in the $i$-th entry of $\vec{a}$.  That is, we decompose a vector $\vec{a}$ that has $\ell$-bit coefficients into an $N$-dimensional vector with 0/1 coefficients.  $\bdifn: \Zq^N \to \Zq^k$ is the inverse of this operation.  Specifically, for a vector $\vec{a} = (a_{1,0}, \ldots, a_{1, \ell-1}, \ldots, a_{k, 0}, \ldots, a_{k, \ell-1}) \in \Zq^N$, we have $$\bdifn(\vec{a}) = \left(\sum 2^j a_{1,j}, \ldots, \sum 2^j a_{k, j}\right)$$  Note that we define this function for any vector $\vec{a} \in \Zq^N$, not just vectors with 0/1 coefficients (which is what $\bdfn$ produces).  Next, for a vector $\vec{a} \in \Zq^N$, define $\flattenfn: \Zq^N \to \Zq^N$ by composing these functions: $$\flattenfn(\vec{a}) = \bdfn(\bdifn(\vec{a}))$$  Finally, define $\powfn: \Zq^k \to \Zq^N$ by $$\powfn(\vec{a}) = (a_1, 2a_1, \ldots, 2^{\ell-1}a_1, \ldots, a_k, 2a_k, \ldots, 2^{\ell-1}a_k)$$  As in GSW, the important properties of these operations are:

\begin{itemize}
\item $\langle \bdfn(\vec{a}), \powfn(\vec{b}) \rangle = \langle \vec{a}, \vec{b} \rangle$
\item $\langle \flattenfn(\vec{a}), \powfn(\vec{b}) \rangle = \langle \vec{a}, \powfn(\vec{b}) \rangle$
\end{itemize}

We extend these operations to matrices by performing them on each row.  Thus, for matrices $A \in \Zq^{N \times k}$, $C \in \Zq^{N \times N}$ and a vector $\vec{b} \in \Zq^k$ we have:

\begin{itemize}
\item $\bdfn(A)\cdot\powfn(\vec{b}) = A\cdot\vec{b}$
\item $\flattenfn(C)\cdot\powfn(\vec{b}) = C\cdot\powfn(\vec{b})$
\end{itemize}

The $\flattenfn$ function keeps our encodings small while maintaining correctness with the zero-tester, which is a result of the $\powfn$ function.  As in GSW, we set $k = (n+1)$ where $n$ is set for security of LWE.

\subsection{GSW encryption}

In this section we define a couple helper procedures that correspond to encryption in GSW.  We set $N = (n+1)\lceil \log q\rceil$ wherever we have parameters $n$ and $q$.

\begin{description}

\item[$(\vec{v}, A) \leftarrow \mathsf{generate}(1^\lambda, n, m, q, \chi):$] For parameters $\lambda, n, m, q \in \mathbb{N}$ and $\chi$, an error distribution on $\Zq^{m\times 1}$, $\mathsf{generate}$ outputs a GSW key pair $(\vec{v},A) \in \Zq^N \times \Zq^{m \times n}$, , by drawing $B \leftarrow \Zq^{m\times n}$, $\vec{t} \leftarrow \Zq^{n \times 1}$, and $\vec{e}\leftarrow \chi$.  Set $\vec{b} = B\vec{t} + \vec{e}$ and $\vec{s} = (1, -t_1, -t_2, \ldots, -t_n)^T \in \Zq^{(n+1)\times m}$.  Set $\vec{v} = \powfn(s) \in \Zq^N$ and $A = \big(\vec{b} \; \big| \; B\big) \in \Zq^{m \times (n+1)}$.  Finally, output $(\vec{v}, A)$.  Note that $A\vec{s} = \vec{b} - B\vec{t} = \vec{e}$.  We use this for decryption.

\item[$C \leftarrow \mathsf{encrypt}(\mu, A,m):$]  For a plaintext $\mu \in \Zq$, a public key $A \in \Zq^{m \times (n+1)}$ as output by the $\mathsf{generate}$ function and dimension $m \in \mathbb{N}$, $\mathsf{encrypt}$ outputs $\flattenfn(\mu I_N + \bdfn(R\cdot A))$ where $R \leftarrow \{0,1\}^{N\times m}$.  When $A$ and $m$ are unambiguous from context we will write $E(\mu) = \mathsf{encrypt}(\mu, A, m)$.

\item[$\mu \leftarrow \mathsf{decrypt}(C, \vec{v}):$]  Given a ciphertext $C$ of a plaintext $\mu$ and a decryption key $\vec{v}$ as output by the $\mathsf{generate}$ function, $\mathsf{decrypt}$ recovers $\mu$ from the product $C\vec{v}$.  
\end{description}

Note that decryption is possible because $\vec{v} = \powfn(\vec{s})$ and $A\vec{s} = \vec{b} - B\vec{t} = \vec{e}$, so by the properties of the bit operations above, $\bdfn(R\cdot A)\vec{v} = R\vec{e}$ for any $R \in \Zq^{N \times m}$.  Thus, as we can write $C = \flattenfn(\mu I_N + \bdfn(R\cdot A))$ for a random binary matrix $R \leftarrow \{0,1\}^{N\times m}$, we have:
\begin{align*}
C\vec{v} &= (\mu I_N + \bdfn(R\cdot A))\vec{v}\\
&= \mu \vec{v} + R\vec{e}
\end{align*}
Thus, as $\vec{e}$ is small and the first $\ell$ coefficients of $\vec{v}$ are of the form $2^i$ for $0 \leq i \leq \ell-1$, we can recover $\mu$.

From \cite{gsw} we have a couple important homomorphic properties, which we summarize in the following lemma:

\begin{lemma}[\cite{gsw}]
\label{gswprop}
Let $(\vec{v}, A) \leftarrow \mathsf{generate}(1^\lambda, n, m, q, \chi)$ for some $\lambda, n, m, q, \chi$.  Then addition and multiplication are homomorphic with respect to $\mathsf{decrypt}$ and $\mathsf{encrypt}$:
\begin{itemize}
\item $\mathsf{decrypt}(E(\mu_1 + \mu_2), \vec{v}) = \mathsf{decrypt}(E(\mu_1) + E(\mu_2), \vec{v})$
\item $\mathsf{decrypt}(E(\mu_1\mu_2), \vec{v}) = \mathsf{decrypt}(E(\mu_1)\cdot E(\mu_2), \vec{v})$
\end{itemize}
\end{lemma}

It is obvious that the first property holds; we briefly recall the proof from \cite{gsw} that shows the second does as well:
\begin{align*}
C_1\cdot C_2\vec{v} &= C_1(\mu_2\vec{v} + \vec{e_2})\\
&= \mu_2(\mu_1\vec{v} + \vec{e_1}) + C_1\vec{e_2}\\
&= \mu_1\mu_2\vec{v} + \mu_2\vec{e_1} + C_1\vec{e_2}
\end{align*}
Clearly, we can also recover $\mu_1\mu_2$ from this as long as $\mu_2$ and the errors $\vec{e_1}, \vec{e_2}$ are all small relative to $q$ because $C_1$ is the result of the $\flattenfn$ function.
