\section{Introduction}

\paragraph{Multilinear maps.} Since the use of bilinear pairings to construct identity-based encryption by Boneh and Franklin \cite{bf} there has been interest in the natural extension to cryptographic maps with multilinearity $\kappa \geq 2$.  Boneh and Silverberg \cite{bs} outlined numerous applications for multilinear maps, but were pessimistic about finding constructions in the field of algebraic geometry.  Since then, there has been some progress in the field.  Garg, Gentry, and Halevi \cite{ggh13a} first proposed the idea of \textit{graded encoding schemes}, which approximate true multilinear maps, and gave a candidate based on ideal lattices (referred to as the GGH construction).  Subsequently, Coron, Lepoint, and Tibouchi gave a similar construction but over the integers (the CLT and CLT15 constructions \cite{clt, clt15}), and Gentry, Gorbunov, and Halevi proposed a construction where the structure of the map is dependent on an underlying acyclic graph \cite{ggh14a}.

Multilinear maps have a range of applications, such as an $N$-way Diffie-Hellman key exchange, broadcast encryption with short keys \cite{bs} or functional encryption (without obfuscation) \cite{blr}.  Arguably the most important and prominent application that motivates the use of multilinear maps is general program obfuscation, originally outlined in \cite{ggh13b} using the GGH scheme, due to its prolific use as a basis for modern cryptographic protocols.

\paragraph{Cryptanalysis of current constructions.}  Unfortunately, the security of these graded encoding schemes is not a well-understood problem, and there has yet to be a candidate that is based on more standard assumptions.  The original CLT and GGH candidates have been successfully attacked \cite{chl, cgh, hj}.  The original CLT scheme was broken completely \cite{chl, cgh}.  The GGH construction has a weak-discrete log attack, and a total break in the particular use-case of key exchange \cite{cgh, hj}.  There have been multiple attempts to fix the CLT scheme and avoid these so-called ``zeroizing attacks" \cite{bwz, clt14, clt15}.  However, only the updated CLT construction \cite{clt15} seems be successful in avoiding these attacks at the cost of some efficiency and increased complexity.  The graph-induced construction in \cite{ggh14a} is known to be insecure in some situations (particularly those involving encodings of zero) through the ``approximate trapdoor" attacks outlined in the original paper, which makes it harder to reason about security in various applications, as it must be proven that those situations do not arise in each specific use of the map.

\paragraph{Our contributions.}  We propose a new multilinear map based on the GSW homomorphic encryption scheme \cite{gsw}.  The primary advantage of our scheme is that sensitive parameters---the zero-tester, zero-tested encodings, encodings of $0$---are either the products of (unknown) uniform matrices or instances of a LWE-esque problem.  In particular, our candidate completely avoids the zeroizing attacks used to attack the CLT and GGH schemes \cite{chl, cgh, hj}.  While a full reduction to standard assumptions does not seem possible at this point, we present our construction both as a step forward in securing multilinear maps under standard assumptions and as an alternative post-zeroizing map to the CLT15 construction.

\paragraph{Future work.}  It remains an open problem to construct multilinear maps based on standard hard problems, or to construct ``true" multilinear maps rather than the graded encoding schemes presented here and in prior works.  Further cryptanalysis of this new scheme could also bring about either greater confidence in its security or additional insight into the general strategy of basing graded encoding schemes on homomorphic encryption schemes as is done in all current constructions; essentially, what security problems arise when zero-testing functionality is added to a HE scheme.  An additional issue with current multilinear map constructions is efficiency.  None of the current candidates, including this one, have reasonable parameters for use in the real world.  Any improvements would have a large impact on the practical viability of flagship applications that rely heaviling on multilinear maps like obfuscation.

Additionally, though there have been several result securing some applications in generic models, the rise of zeroizing attacks has questioned the validity of these assumptions.  It would be useful to prove certain protocols secure under weaker assumptions, such as those presented here.

\paragraph{Overview of the GSW HE scheme.}  The encoding of elements and homomorphic operations in our scheme borrow heavily from the GSW homomorphic encryption scheme.  We recall the main details of the scheme in broad strokes here.  Ciphertexts in GSW look like the following, where $\mu \in \mathbb{Z}_q$ is the plaintext message and $A$ is the public key (a ``uniform" matrix) with private key $\vec{s}$ (a vector such that $A\cdot \vec{s} = \vec{e}$, a small vector) hiding the plaintext with some noise $R$ (a binary matrix):

$$C = \flattenfn(\mu I_N + \bdfn(R\cdot A))$$

We will define the $\flattenfn$ and $\bdfn$ functions more precisely later, but for now we simply observe that for any matrix $B$, $\flattenfn(B)$ is always a matrix with 0/1 entries, $\flattenfn(B)\cdot \vec{v} = B\vec{v}$, and $\bdfn(B)\cdot \vec{v} = B\vec{s}$ where $\vec{v}$ is a particular type of vector that exhibits these properties.  More precisely, we have:
\begin{align*}
C\cdot \vec{v} &= \flattenfn(\mu I_N + \bdfn(R\cdot A))\cdot \vec{v}\\
&= (\mu I_N + \bdfn(R\cdot A))\cdot \vec{v}\\
&= \mu \vec{v} + R\cdot A \cdot \vec{s} \\
&= \mu \vec{v} + R\cdot \vec{e}\\
&= \mu \vec{v} + error
\end{align*}
The plaintext $\mu$ can then be recovered because $R\cdot \vec{e}$ is small.    

\paragraph{Getting a zero-testing element.}
The main thing to note about the GSW scheme is that the decryption key $\vec{v}$ exhibits some interesting properties that allow us some asymmetry with respect to ciphertexts, plaintexts, and noise growth.  To illustrate, we recall why GSW ciphertexts are multiplicatively homomorphic (with respect to the decryption key):
\begin{align*}
C_1C_2\cdot \vec{v} &= C_1\cdot (\mu_2 \vec{v} + \vec{e_2})\\
&= \mu_2\mu_1 \vec{v} + \mu_2\vec{e_1}+ C_1\vec{e_2}
\end{align*}
Note how the resulting noise factor is dependent on the plaintext $\mu_2$ and the error in the first ciphertext $\vec{e_1}$.  This is a nice optimization that can help keep error terms small if $\mu_2 = 0$; however, we can exploit this property ourselves by creating a ciphertext where the error $\vec{e_1}$ is large (uniformly random).  That way, if our plaintext $\mu_2 = 0$, then the result properly ``decrypts," whereas if $\mu_2 = 0$ then the result will be garbage and unusable to an adversary, thus giving us a zero-testing element for the scheme.  To do this without breaking certain necessary hardness assumptions for multilinear maps we need to accomplish two things: (a) hide the decryption key and (b) require that a user multiply the ciphertext $C$ on the left by our ``malformed ciphertext" $C'$ that has ``too much noise."  We do this by generating a private random matrix $T$ and changing our encodings to be $C = TC_\mu T^{-1}$, where $C_\mu$ is the original GSW ciphertext.  We can then take our malformed ciphertext to be $C' = \bdfn(U)\cdot T^{-1}$, for a uniformly random $U$, and our hidden decryption key to be $T(\vec{v} + \vec{\epsilon})$ for a small vector $\vec{\epsilon}$.  This gives us the desired properties:
$$C'\cdot C \cdot T(\vec{v} + \vec{\epsilon}) = \mu (U\vec{s}) + \vec{e}_C + C_\mu\vec{\epsilon}$$
Of course, this is simply a starting point; for example, we also introduce the idea of levels through an element $z$, as in other multilinear schemes, so that we can only zero-test at a specific level-$\kappa$. 
