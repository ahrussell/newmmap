\section{Introduction}

\paragraph{Multilinear maps.} Since the use of bilinear pairings to construct identity-based encryption by Boneh and Franklin \cite{bf} there has been interest in the natural extension to cryptographic maps with multilinearity $\kappa \geq 2$.  Boneh and Silverberg \cite{bs} outlined numerous applications for multilinear maps, but were pessimistic about finding constructions in the field of algebraic geometry.  Since then, there has been some progress in the field.  Garg, Gentry, and Halevi \cite{ggh13a} first proposed the idea of \textit{graded encoding schemes}, which approximate true multilinear maps, and gave a candidate based on ideal lattices (referred to as the GGH construction).  Subsequently, Coron, Lepoint, and Tibouchi gave a similar construction but over the integers (the CLT construction \cite{clt, clt15}), and Gentry, Gorbunov, and Halevi proposed a construction where the structure of the map is dependent on an underlying acyclic graph \cite{ggh14a}.

Multilinear maps have a range of applications, such as an $N$-way Diffie-Hellman key exchange \cite{bs} or functional encryption (without obfuscation) \cite{blr}.  Arguably the most important and prominent application that motivates the use of multilinear maps is general program obfuscation, originally outlined in \cite{ggh13b} using the GGH scheme, due to its prolific use as a cryptographic primitive.

\paragraph{Cryptanalysis of current constructions.}  Unfortunately, the security of these graded encoding schemes is not a well-understood problem, and there has yet to be a candidate that is based on more standard assumptions.  The original CLT and GGH candidates have been successfully attacked \cite{chl, cgh, hj}.  The original CLT scheme was broken completely \cite{chl, cgh}.  The GGH construction has a weak-discrete log attack, and a total break in the particular use-case of key exchange \cite{cgh, hj}.  There have been multiple attempts to fix the CLT scheme \cite{bwz, clt14, clt15}.  However, only the updated CLT construction \cite{clt15} seems to avoid these attacks at the cost of some efficiency.  The graph-induced construction in \cite{ggh14a} is known to be insecure in some situations through the ``approximate trapdoor" attacks outlined in the original paper, which makes it harder to reason about security in various applications, as you must prove that those situations do not arise in each specific use of the map.

\paragraph{Our contributions.}  We propose a new multilinear map based on a variant of the GSW homomorphic encryption scheme \cite{gsw}.  The primary advantage of our scheme is that its security is based on variants of the RLWE problem.  While a full reduction does not seem possible at this point, our construction relies on assumptions that differ dramatically from those of prior constructions and avoids prior vulnerabilities.  Our construction also provides the additional advantage that arbitrary plaintext elements can be encoded, unlike in other constructions where the plaintext of a level-0 encoding is unknown to the user and must be sampled from some hidden distribution.  In fact, this sampling process has been a source of attacks in the first CLT construction \cite{chl, cgh}.   

\paragraph{Future work.}  It remains an open problem to construct multilinear maps based on standard hard problems, or to construct ``true" multilinear maps rather than the graded encoding schemes presented here and in prior works.  Further cryptanalysis of this new scheme could also bring about either greater confidence in its security or additional insight into the general strategy of basing graded encoding schemes on homomorphic encryption schemes as is done in all current constructions; essentially, what security problems arise when zero-testing functionality is added to a HE scheme.  An additional issue with current multilinear map constructions is efficiency.  None of the current candidates, including this one, have reasonable parameters for use in the real world.  Any improvements would have a large impact on the practical viability of flagship applications like obfuscation that rely heavily on multilinear maps.

\paragraph{Overview of the GSW HE scheme.}  The encoding of elements and homomorphic operations are exactly as in the GSW HE scheme, except we work over a polynomial ring $\Rq = \mathbb{Z}_q[x] / \left<x^n + 1\right>$ instead of $\mathbb{Z}_q$.  To be more specific, ciphertexts in GSW look like the following, where $\mu \in \mathbb{Z}_q$ is the plaintext message and $A$ is the public key (a ``uniform" matrix) hiding the plaintext with some noise $R$ (a binary matrix):

$$C = \mu I_N + R\cdot A$$

In GSW the decryption key $v$ (a vector) acts as an ``approximate inverse" to the public key $A$, so that $A\cdot v = e$ where $e$ is small.  Thus when we multiply $C$ by $v$ we get the following:
\begin{equation}
\label{gswintro}
C\cdot v = \mu v + R\cdot e
\end{equation}
The plaintext $\mu$ can then be recovered because $R\cdot e$ is small.    

\paragraph{Getting a zero-testing element.}
The main thing to note in \eqref{gswintro} is that the GSW decryption key is itself ``large," so that when we apply it to a ciphertext the result will be large if and only if $\mu$ is non-zero.  Thus, if we can somehow securely publish $v$ we can then zero-test encodings, which gives us a graded encoding scheme.  A starting point for hiding the key is that ciphertexts are themselves small (binary matrices), as they are a result of the $\flattenfn$ function, a helpful operation that allows noise growth to be bounded in the original GSW scheme.  This way, we can add a ``small" value $\epsilon$ to the decryption key and still get the desired property:
\begin{align*}
C\cdot(v + \epsilon) &= \mu v + R\cdot e + C\cdot \epsilon \\
&= \mu v + small
\end{align*}
Of course, this does not fully hide $v$ nor does it make decryption any harder.  So, we multiply $v$ by a ``small" value $\alpha$, which brings us closer:
\begin{align*}
C\cdot(\alpha v + \epsilon) &= \mu (\alpha v) + \alpha R\cdot e + C\cdot \epsilon \\
&= \mu (\alpha v) + small
\end{align*}
This breaks decryption in the case that $\alpha$ is unknown, but still maintains the property that the result will be small if and only if $\mu$ is small, so zero-testing will still work.