\section{Our construction}

We are now ready to outline our map, which is an instantiation of a graded encoding scheme as in prior works \cite{clt, ggh13a, clt15}.  The basic operations (setup, encoding, addition, multiplication) of our scheme are essentially identical to those of the GSW scheme.

In general, a fresh encoding of a plaintext $\mu \in \Rq$ at level-$i$ will look like the following, where $z$ is a random ring element, $T$ is a random matrix, and $A$ is a GSW public key:
$$C = z^{-i}T \cdot E(\mu)\cdot T^{-1}$$
recalling that $E(\mu) = \flattenfn\left(\mu I_{N} + \bdfn(R\cdot A))\right)$ is a GSW encryption of $\mu$. Thus, encodings are simply GSW ciphertexts (which are small, as they are a result of the $\flattenfn$ function) pre/post-multiplied by a random matrix $T$ and its inverse and by the ring element $z^{-i}$ to introduce the notion of levels (and we work over a polynomial ring instead of the integers).  For an encoding $C$ we will denote the corresponding GSW encryption of $\mu$ by $C_\mu$.

\paragraph{Setup: $(\mathsf{pp},\pzt) \leftarrow \mathsf{setup}(1^\lambda, 1^\kappa)$.}  Given the security parameter $\lambda$ and the multilinearity level $\kappa$, we generate the public parameters and the zero-testing element.  The parameters $q$ and $n$ define the ring $\Rq$.  Set $m = O(\log q)$ and $N = 2\ell$ where $\ell = \lceil \log q \rceil$.  Let $\chi$ be an error distribution on $\Rq$.  $\sigma$ is a parameter for sampling plaintext elements from a discrete Gaussian distribution.  We generate the GSW key pair $(\vec{v},A) \leftarrow\mathsf{generate}(1^\lambda,m, \chi)$.  We generate the {\bf zero-testing parameter} $\pzt$ by first drawing uniform matrices $U \leftarrow \Rq^{N\times 2}$ and $T \leftarrow \Rq^{N \times N}$ and setting $$\w = (1,1,\ldots,1)\cdot \bdfn(U)\cdot T^{-1}$$  Then draw a uniform element $z \leftarrow \Rq$ and then a few ``somewhat small" elements $\alpha \leftarrow \mathcal{D}_{\mathbb{Z}^n, \sigma_\alpha}$, $\vec{\epsilon} \leftarrow (\mathcal{D}_{\mathbb{Z}^n, \sigma_\epsilon})^N$, and $\vec{\delta} \leftarrow (\mathcal{D}_{\mathbb{Z}^n, \sigma_\delta})^N$ for some parameters $\sigma_\alpha, \sigma_\delta,$ and $\sigma_\epsilon$.  We then set $$\u = z^\kappa T\cdot (\alpha(\vec{v} + \vec{\delta}) + \vec{\epsilon})$$  Intuitively, $\u$ is a (R)LWE ``encryption" of $\vec{v}$, the original decryption key in GSW (plus some other noise that we need).  We generate a level-1 encoding $Y$ by drawing $R \leftarrow \{0,1\}^{N\times m}$ and setting $$Y = T\cdot\flattenfn(I_N + \bdfn(R\cdot A))\cdot T^{-1}$$  which is a level-1 encoding of 1. We generate the rerandomization parameters $X_i$ by drawing $R_i \leftarrow \{0,1\}^{N \times m}$ and setting $$X_i = z^{-i}T\cdot\flattenfn(\bdfn(R_i\cdot A))\cdot T^{-1}$$ for $1 \leq i \leq \tau$.  Note that these are level-1 encodings of 0. We output $\mathsf{pp} = (\kappa, q,n,m,N,\sigma, Y, \{X_i\}_{i=1}^\tau)$, $P$, and $\pzt = (\w,\u)$.

\paragraph{Zero testing:  $\mathsf{isZero}(C, \pzt) \stackrel{?}{=} 0/1$.}  To zero-test a (fresh) encoding $C = z^{-\kappa} T C_\mu T^{-1}$ of $\mu$ at level-$\kappa$, output 1 if $\norm{\infty}{\left<\w, C\u\right>} < P$ and 0 otherwise.  To see briefly why this is correct, recall the properties of the bit operations and GSW decryption from Section 2.  Then we have:
\begin{align*}
C\u &= z^{-\kappa} T C_\mu T^{-1} \cdot \u\\
&= z^{-\kappa} T C_\mu  T^{-1} \cdot z^\kappa T\cdot ((\alpha(z^\kappa\vec{v} + \vec{\delta}) + \vec{\epsilon})\\
&= (z^{-\kappa} z^{\kappa}) T \cdot C_\mu (\alpha(\vec{v} + \vec{\delta}) + \vec{\epsilon})\\
&= T \cdot (\alpha(C_\mu\vec{v} + C_\mu\vec{\delta}) + C_\mu\vec{\epsilon})\\
&= T \cdot (\alpha\left(\flattenfn(\mu I_{N} + \bdfn(R\cdot A))\vec{v} + C_\mu\vec{\delta}\right) + C_\mu\vec{\epsilon})\\
&= T \cdot (\alpha(\mu \vec{v} + \vec{e}) + C_\mu\vec{\delta}) + C_\mu\vec{\epsilon})
\end{align*}
and
\begin{align*}
\left<\w, C\u\right> &= (1,1,\ldots,1)\cdot\bdfn(U)\cdot T^{-1}\cdot T \cdot (\alpha((\mu \vec{v} + \vec{e}) + C_\mu\vec{\delta}) + C_\mu\vec{\epsilon})\\
&= (1,1,\ldots,1)\cdot\bdfn(U) \cdot (\alpha((\mu \vec{v} + \vec{e}) + C_\mu\vec{\delta}) + C_\mu\vec{\epsilon})\\
&= (1,1,\ldots,1)\cdot(\mu(\alpha U \vec{v}) + \bdfn(U)\cdot (\alpha\vec{e} + C_\mu(\alpha\vec{\delta} + \vec{\epsilon})))\\
&= (1,1,\ldots,1)\cdot(\mu(big) + small)
\end{align*} 
Intuitively, note that this element is large if and only if $\mu \not= 0$ because all elements are small except for $\vec{v}$ and $U$.  Formally, we prove the following lemma in the appendix.

\begin{lemma}
\label{zerotesting}
This is the lemma that formally outlines the size difference between a zero and nonzero encoding (informally stated for now).  Let $C = z^{-\kappa} TC_\mu T^{-1}$ where $C_\mu$ is a GSW encryption of $\mu \in \Rq$ (i.e., $C_\mu\vec{v}$ decrypts to $\mu$).  Then $\norm{\infty}{\left<\w, C\u\right>} < P$ if and only if $\mu = 0$.
\end{lemma}

\paragraph{Addition \& subtraction.}  For two encodings $C_1$ and $C_2$ at the same level-$i$, addition and subtraction are the corresponding matrix operations. As $C_{\mu_1}+ C_{\mu_2}$ is a valid GSW ciphertext of $\mu_1 + \mu_2$ by Lemma \ref{gswprop}, the correctness with respect to zero-testing for $C_1 + C_2 = z^{-i}T(C_{\mu_1}+C_{\mu_2})T^{-1}$ follows from Lemma \ref{zerotesting}.  Subtraction follows similarly.

\paragraph{Multiplication.}  Multiplication in our scheme is also matrix multiplication.  Suppose that $C_1$ and $C_2$ are encodings at level-$i$ and level-$j$, respectively, with $i+j \leq \kappa$ and corresponding GSW ciphertexts $C_{\mu_1}$ and $C_{\mu_2}$.  First note that
\begin{align*}
C_1\cdot C_2 &= (z^{-i}TC_{\mu_1}T^{-1})\cdot(z^{-j}TC_{\mu_2}T^{-1})\\
&= z^{-i-j}T(C_{\mu_1}\cdot C_{\mu_2})T^{-1}
\end{align*}
which is a level-$(i+j)$ encoding of $\mu_1\mu_2$ by Lemma \ref{zerotesting}, because $C_{\mu_1} \cdot C_{\mu_2}$ correctly decrypts to $\mu_1\mu_2$, by Lemma \ref{gswprop}, so long as $\mu_2$ is small---for this reason our message space is restricted, as in CLT and GGH (to control noise growth).

\paragraph{Sampling:  $\mu \leftarrow \mathsf{samp}(\mathsf{pp})$.}  To sample a level-0 encoding---that is, a plaintext element---simply draw $\mu\leftarrow \mathcal{D}_{\mathbb{Z}^n, \sigma}$.

\paragraph{Rerandomization: $C' \leftarrow \mathsf{rerand}(\mathsf{pp}, C)$.}  To rerandomize a level-1 encoding $C$, draw $r_j \leftarrow \{0,1\}$ for $0 \leq j \leq \tau$ and compute $$C' = C + \sum_{j=1}^\tau r_jX_j$$ using the addition operator of the scheme.

\paragraph{Encoding:  $C \leftarrow \mathsf{enc}(\mathsf{pp},U,i)$.}  A user can encode a plaintext $\mu$ to level-$i$ by first computing $Y' \leftarrow \mathsf{rerand}(\mathsf{pp}, Y)$, which will be a rerandomized level-1 encoding of 1, and then setting $$C = \mu \cdot \left(Y'\right)^i$$ using the multiplication operator of the scheme, which will be a level-$i$ encoding of $\mu$.

\paragraph{Extraction: $sk \leftarrow \mathsf{ext}(\mathsf{pp}, \pzt, C^{(\kappa)})$.}  As in \cite{ggh13a, clt15} we apply the zero-tester to a level-$\kappa$ encoding $C^{(\kappa)}$ and collect the most significant bits to extract a random function of the underlying plaintext $\mu$.

\begin{lemma}
\label{extraction}
This is the lemma that states we can extract a random function of $\lambda$ bits from a level-$\kappa$ encoding.
\end{lemma}





\subsection{Asymmetric variants}

As in CLT and GGH we can instead compute $z_i$ for $i \in [\kappa]$ and encode plaintexts to specific index sets $S \subseteq [\kappa]$ rather than levels.  This is required for some applications like obfuscation.  We can also break the commutativity of the scheme and enforce a particular order of multiplication by generating many matrices $T_i$ rather than a single matrix $T$.

\subsection{Setting parameters}

As in homomorphic encryption schemes and multilinear maps, the size of the noise at level-$\kappa$ is dependent on the size of the initial error $B_e$, the multilinearity level $\kappa$, and the initial size of plaintexts $B_\mu$.  Specifically, we have the noise bounded by:
\begin{equation}
\label{noise1}
B_\kappa = (nN\cdot B_\mu)^\kappa \cdot B_e
\end{equation}

Then, when we apply the zero-testing element at level-$\kappa$ we have:
\begin{equation}
\label{noise2}
C\pzt = \alpha (\mu \vec{v} + \vec{e_{\circ}} + C\vec{\delta}) + C\vec{\epsilon}
\end{equation}
where the size of $e_{\circ}$ is bounded by $B_\kappa$, as in \eqref{noise1}.  In the case of $\mu = 0$ we then have the size of \eqref{noise2} bounded by $B_\alpha(B_\kappa + nNB_\delta) + nNB_\epsilon$.

\paragraph{Parameters that need to be set.}

\begin{itemize}
\item $n$, the degree of polynomials.  Dependent on RLWE security.
\item $q$, the size of the underlying field.  Dependent on RLWE security and (more importantly) the size of zero-tested elements, which depends on $n, N, \chi, \kappa, \sigma, \sigma_\alpha, \sigma_\delta, \sigma_\epsilon$.
\item $\sigma$, the parameter for the Gaussian that samples plaintexts.  GGH bounds the size of their ciphertexts (not their plaintexts by $q^{1/8}$).
\item $\sigma_\alpha, \sigma_\delta, \sigma_\epsilon$, the parameters for the Gaussians that draw noise elements for the zero-tester.  GGH sets $\sigma_\alpha$ to be $\sqrt{q}$.  Should be ``as large as possible" but obviously cannot yield a uniform distribution.
\item $\chi$, the error distribution for the RLWE instance $A$.
\item $\tau$, the number of encodings of $0$ we publish for rerandomization.
\end{itemize}


