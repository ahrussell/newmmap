\section{Our construction}

We are now ready to outline our map, which is an instantiation of a graded encoding scheme as in prior works \cite{clt, ggh13a, clt15}.  Refer to Appendix A for an explicit definition of graded encoding schemes. 

In general, a fresh encoding of a plaintext $\mu \in \Zq$ at level-$i$ will look like the following, where $z \leftarrow \Zq$ and $T\leftarrow \Zq^{N \times N}$:
$$C = z^{-i}T \cdot E(\mu)\cdot T^{-1}$$
recalling that $E(\mu) = \flattenfn\left(\mu I_{N} + \bdfn(R\cdot A))\right)$ is a GSW encryption of $\mu$. Thus, encodings are simply GSW ciphertexts (which are small, as they are a result of the $\flattenfn$ function) pre/post-multiplied by a random matrix $T$ and its inverse and by the parameter $z^{-i}$ to introduce the notion of levels.  For an encoding $C$ of a plaintext $\mu$ we will denote the corresponding GSW encryption by $C_\mu$.  Let $\vec{1}$ denote the row vector $(1,1,\ldots, 1) \in \Zq^{1 \times N}$.

\paragraph{Setup: $(\mathsf{pp},\pzt) \leftarrow \mathsf{setup}(1^\lambda, 1^\kappa)$.}  Given the security parameter $\lambda$ and the multilinearity level $\kappa$, we generate the public parameters and the zero-testing element.  Let $n,q, \chi$ be such that the LWE$_{n,q,\chi}$ assumption holds for a security level of $\lambda$.  Set $m = O(\log q)$ and $N = (n+1)\ell$ where $\ell = \lceil \log q \rceil$.  Let $\chi$ be an error distribution on $\Zq$.  $\sigma$ is a parameter for sampling plaintext elements from a discrete Gaussian distribution.  We generate the GSW key pair $(\vec{v},A) \leftarrow \mathsf{generate}(1^\lambda,n,m, \chi)$.  We generate the zero-testing parameter $\pzt$ by first drawing uniform matrices $U \leftarrow \Zq^{N\times (n+1)}$ and $T \leftarrow \Zq^{N \times N}$ and setting $$\w = \vec{1}\cdot \bdfn(U)\cdot T^{-1}$$  Then draw a uniform element $z \leftarrow \Zq$ and a ``somewhat small" element $\vec{\epsilon} \leftarrow (\mathcal{D}_{\mathbb{Z}, \sigma'})^N$ for a parameter $\sigma'$.  We then set $$\u = z^\kappa T\cdot (\vec{v} + \vec{\epsilon})$$  We generate a level-1 encoding of 1 $$Y = z^{-1}T\cdot E(1) \cdot T^{-1}$$  We generate the rerandomization parameters $$X_i = z^{-1}T\cdot E(0) \cdot T^{-1} \quad \forall i \in [\tau]$$  Note that these are level-1 encodings of 0. We output $\mathsf{pp} = (\kappa, q,N,\sigma, Y, \{X_i\}_{i=1}^\tau)$, $P$, and $\pzt = (\w,\u)$.

\paragraph{Zero testing:  $\mathsf{isZero}(C, \pzt) \stackrel{?}{=} 0/1$.}  To zero-test an encoding $C = z^{-\kappa} T C_\mu T^{-1}$ of $\mu$ at level-$\kappa$, output 1 if $\norm{\infty}{\left<\w, C\u\right>} < P$ and 0 otherwise.  To see briefly why this is correct, recall the properties of the bit operations and GSW decryption from Section \ref{sec:prelim}.  Then we have the following, where $\vec{e}_C$ is the error present in the GSW encryption $C_\mu$:
\begin{align*}
C\u &= z^{-\kappa} T C_\mu T^{-1} \cdot \u\\
&= z^{-\kappa} T C_\mu  T^{-1} \cdot z^\kappa T\cdot (\vec{v} + \vec{\epsilon})\\
&= (z^{-\kappa} z^{\kappa}) T \cdot C_\mu (\vec{v} + \vec{\epsilon})\\
&= T \cdot (C_\mu\vec{v} + C_\mu\vec{\epsilon})\\
&= T \cdot (\mu \vec{v} + \vec{e}_C + C_\mu\vec{\epsilon})
\end{align*}
which gives us
\begin{align*}
\left<\w, C\u\right> &= \vec{1}\cdot\bdfn(U)\cdot T^{-1}\cdot T \cdot (\mu \vec{v} + \vec{e}_C + C_\mu\vec{\epsilon})\\
&= \vec{1}\cdot\bdfn(U) \cdot (\mu \vec{v} + \vec{e}_C + C_\mu\vec{\epsilon})\\
&= \vec{1}\cdot(\mu(U \vec{s}) + \bdfn(U)\cdot (\vec{e}_C + C_\mu\vec{\epsilon}))\\
&= \vec{1}\cdot(\mu(big) + small)
\end{align*} 
Intuitively, note that this element is large if and only if $\mu \not= 0$ because all elements are small except for $\vec{v}$ and $U$.  Formally, we prove the following lemma in the appendix.

\begin{lemma}
\label{zerotesting}
This is the lemma that formally outlines the size difference between a zero and nonzero encoding (informally stated for now).  Let $C = z^{-\kappa} TC_\mu T^{-1}$ where $C_\mu$ is a GSW encryption of $\mu \in \Zq$ (i.e., $C_\mu\vec{v}$ decrypts to $\mu$).  Then $\norm{\infty}{\left<\w, C\u\right>} < P$ if and only if $\mu = 0$.
\end{lemma}

\paragraph{Addition \& subtraction.}  For two encodings $C_1$ and $C_2$ at the same level-$i$, addition and subtraction are the corresponding matrix operations. As $C_{\mu_1}+ C_{\mu_2}$ is a valid GSW ciphertext of $\mu_1 + \mu_2$ by Lemma \ref{gswprop}, the correctness with respect to zero-testing for $C_1 + C_2 = z^{-i}T(C_{\mu_1}+C_{\mu_2})T^{-1}$ follows from Lemma \ref{zerotesting}.  Subtraction follows similarly.

\paragraph{Multiplication.}  Multiplication in our scheme is also matrix multiplication.  Suppose that $C_1$ and $C_2$ are encodings at level-$i$ and level-$j$, respectively, with $i+j \leq \kappa$ and corresponding GSW ciphertexts $C_{\mu_1}$ and $C_{\mu_2}$.  First note that
\begin{align*}
C_1\cdot C_2 &= (z^{-i}TC_{\mu_1}T^{-1})\cdot(z^{-j}TC_{\mu_2}T^{-1})\\
&= z^{-i-j}T(C_{\mu_1}\cdot C_{\mu_2})T^{-1}
\end{align*}
which is a level-$(i+j)$ encoding of $\mu_1\mu_2$ by Lemma \ref{zerotesting}, because $C_{\mu_1} \cdot C_{\mu_2}$ correctly decrypts to $\mu_1\mu_2$, by Lemma \ref{gswprop}, so long as $\mu_2$ is small---for this reason our message space is restricted, as in CLT and GGH (to control noise growth).

\paragraph{Sampling:  $\mu \leftarrow \mathsf{samp}(\mathsf{pp})$.}  To sample a level-0 encoding---that is, a plaintext element---simply draw $\mu\leftarrow \mathcal{D}_{\mathbb{Z}^n, \sigma}$.

\paragraph{Rerandomization: $C' \leftarrow \mathsf{rerand}(\mathsf{pp}, C)$.}  To rerandomize a level-1 encoding $C$, draw $r_j \leftarrow \{0,1\}$ for $0 \leq j \leq \tau$ and compute $$C' = C + \sum_{j=1}^\tau r_jX_j$$ using the addition operator of the scheme.

\paragraph{Encoding:  $C \leftarrow \mathsf{enc}(\mathsf{pp},U,i)$.}  A user can encode a plaintext $\mu$ to level-$i$ by first computing $Y' \leftarrow \mathsf{rerand}(\mathsf{pp}, Y)$, which will be a rerandomized level-1 encoding of 1, and then setting $$C = \mu \cdot \left(Y'\right)^i$$ using the multiplication operator of the scheme, which will be a level-$i$ encoding of $\mu$.

\paragraph{Extraction: $sk \leftarrow \mathsf{ext}(\mathsf{pp}, \pzt, C^{(\kappa)})$.}  As in \cite{ggh13a, clt15} we apply the zero-tester to a level-$\kappa$ encoding $C^{(\kappa)}$ and collect the most significant bits to extract a random function of the underlying plaintext $\mu$.

\begin{lemma}
\label{extraction}
This is the lemma that states we can extract a random function of $\lambda$ bits from a level-$\kappa$ encoding.
\end{lemma}

\subsection{Asymmetric variants}

As in CLT and GGH we can instead compute $z_i$ for $i \in [\kappa]$ and encode plaintexts to specific index sets $S \subseteq [\kappa]$ rather than levels.  This is required for some applications like obfuscation.  We can also break the commutativity of the scheme and enforce a particular order of multiplication by generating many matrices $T_i$ rather than a single matrix $T$.

\subsection{Setting parameters}

As in homomorphic encryption schemes and multilinear maps, the size of the noise at level-$\kappa$ is dependent on the size of the initial error $B_e$, the multilinearity level $\kappa$, and the initial size of plaintexts $B_\mu$.  Specifically, we have the noise bounded by:
\begin{equation}
\label{noise1}
B_\kappa = (nN\cdot B_\mu)^\kappa \cdot B_e
\end{equation}

Then, when we apply the zero-testing element at level-$\kappa$ we have:
\begin{equation}
\label{noise2}
\vec{1}\cdot(\mu(U \vec{s}) + \bdfn(U)\cdot (\vec{e}_C + C_\mu\vec{\epsilon}))\\
\end{equation}
where the size of $\vec{e}_C$ is bounded by $B_\kappa$, as in \eqref{noise1}.  In the case of $\mu = 0$ we then have the size of \eqref{noise2} bounded by 

\paragraph{Parameters that need to be set.}

\begin{itemize}
\item $n$, the size of the LWE secrets.  Dependent on RWE security.
\item $q$, the size of the underlying field.  Dependent on RLWE security and (more importantly) the size of zero-tested elements, which depends on $n, N, \chi, \kappa, \sigma, \sigma'$.
\item $\sigma$, the parameter for the Gaussian that samples plaintexts.  GGH bounds the size of their ciphertexts (not their plaintexts by $q^{1/8}$).
\item $\sigma'$, the parameter for the Gaussians that draw a noise element for the zero-tester.  Should be ``as large as possible" but obviously cannot yield a uniform distribution.
\item $\chi$, the error distribution for the LWE instance $A$.
\item $\tau$, the number of encodings of $0$ we publish for rerandomization.
\end{itemize}


